\documentclass[11pt]{article}
\usepackage[utf8]{inputenc}
\usepackage{hyperref}
\setlength{\topmargin}{-.5in}
\setlength{\textheight}{9in}
\setlength{\oddsidemargin}{.125in}
\setlength{\textwidth}{6.25in}
\begin{document}
\title{"Crystal Maze", Documentation Graphiste}
\author{Thomas Bombrun, Dimitri Bouleau, Thomas Coeffic, Caroline Ramond, Th�o Van-Lede\\
INPG - Ensimag}
\renewcommand{\today}{17 Juin 2014}
\maketitle

Ce document est destin� � un graphiste qui souhaiterait am�liorer le jeu "Crystal Maze" d'un point de vue graphique. Il pr�sente les d�marches � suivre pour modifier les ressources visuelles du jeu.

%--------------------------------
\section{Arborescence du projet}
Seul le dossier $www$ contient les fichiers � modifier. Les ressources visuelles $.png$ se trouvent dans le dossier $www/ressources$. Les ressources correspondant � l'aide et aux tutoriels sont �galement pr�sentes au format $.xcf$ dans le dossier $www/ressource-dev$ afin d'�tre plus facilement modifiables. Vous pouvez �galement �tre amen�s � modifier le fichier $www/conf.yaml$ dans le cas o� vous d�sireriez cr�er de nouvelles ressources plut�t que d'�craser celles existantes. C'est �galement ce fichier qu'il faudra modifier dans le cas o� vous voudriez modifier les dimensions des ressources.

%--------------------------
\section{Ecraser/Ajouter une ressource $.png$}
Il est conseill� d'�craser les anciennes ressources plut�t que d'en cr�er de nouvelles, afin de limiter le temps de chargement du jeu. C'est �galement la solution la plus simple puisqu'elle n'exige aucune manipulation suppl�mentaire de votre part. Si toutefois vous souhaitez cr�er de nouvelles ressources, il faudra modifier le fichier de configuration $www/conf.yaml$. Ce fichier contient, entre autres, les adresses de toutes les ressources utilis�es pour le jeu sous la forme suivante : \\
\begin{verbatim}
RessourceUrl : adresse_de_la_ressource
\end{verbatim}
Ainsi, il suffit de changer l'adresse de la ressource dans ce fichier pour que la nouvelle ressource remplace l'ancienne.

%--------------------------
\section{Modifier la taille des ressources}
Il est fortement conseill� de garder les dimensions des ressources d'origine au risque de ne plus avoir des boutons/�crans centr�s. Dans le cas o� vous voudriez changer les dimensions des images, il sera n�cessaire de changer les constantes du fichier $www/conf.yaml$. Par exemple, pour changer la taille des boutons "Next Level", "Play" etc, il faudra modifier les champs "BUTTON_WIDTH" et "BUTTON_HEIGHT" du fichier $.yaml$.

%--------------------------
\section{Ressources xcf}
Il s'agit de ressources d�pendantes d'autres ressources : �crans d'aide et tutoriels. Par exemple, le menu d'aide concernant le bloc simple devra �tre modifi� si l'image du bloc simple est modifi�e. 

%---------------------------
\section{Attention : Compatibilit� mobile}
Pour que vos modifications soient effectives sur les applications mobiles, il faut recompiler toute l'application. Pour ce faire, r�f�rez-vous � la documentation technique.

\end{document}

