\documentclass[11pt, english]{article}
\usepackage{babel}
\usepackage[utf8]{inputenc}
\usepackage{hyperref}
\usepackage[babel=true]{csquotes}
\setlength{\topmargin}{-.5in}
\setlength{\textheight}{9in}
\setlength{\oddsidemargin}{.125in}
\setlength{\textwidth}{6.25in}
\begin{document}
\title{\enquote{Crystal Maze}, Documentation Graphiste}
\author{Thomas Bombrun, Dimitri Bouleau, Thomas Coeffic, Caroline Ramond, Théo Van-Lede\\
INPG - Ensimag}
\renewcommand{\today}{17 Juin 2014}
\maketitle

Ce document est destiné à un graphiste qui souhaiterait améliorer le jeu \enquote{Crystal Maze} d'un point de vue graphique. Il présente les démarches à suivre pour modifier les ressources visuelles du jeu.

%--------------------------------
\section{Arborescence du projet}
Seul le dossier $www$ contient les fichiers à modifier. Les ressources visuelles $.png$ se trouvent dans le dossier $www/ressources$. Les ressources correspondant à l'aide et aux tutoriels sont également présentes au format $.xcf$ dans le dossier $www/ressource-dev$ afin d'être plus facilement modifiables. Vous pouvez également être amenés à modifier le fichier $www/conf.yaml$ dans le cas où vous désireriez créer de nouvelles ressources plutôt que d'écraser celles existantes. C'est également ce fichier qu'il faudra modifier dans le cas où vous voudriez modifier les dimensions des ressources.

%--------------------------
\section{Ecraser/Ajouter une ressource $.png$}
Il est conseillé d'écraser les anciennes ressources plutôt que d'en créer de nouvelles, afin de limiter le temps de chargement du jeu. C'est également la solution la plus simple puisqu'elle n'exige aucune manipulation supplémentaire de votre part. Si toutefois vous souhaitez créer de nouvelles ressources, il faudra modifier le fichier de configuration $www/conf.yaml$. Ce fichier contient, entre autres, les adresses de toutes les ressources utilisées pour le jeu sous la forme suivante : \\
\begin{verbatim}
RessourceUrl : adresse_de_la_ressource
\end{verbatim}
Ainsi, il suffit de changer l'adresse de la ressource dans ce fichier pour que la nouvelle ressource remplace l'ancienne.

%--------------------------
\section{Modifier la taille des ressources}
Il est fortement conseillé de garder les dimensions des ressources d'origine au risque de ne plus avoir des boutons/écrans centrés. Dans le cas où vous voudriez changer les dimensions des images, il sera nécessaire de changer les constantes du fichier $www/conf.yaml$. Par exemple, pour changer la taille des boutons \enquote{Next Level}, \enquote{Play} etc, il faudra modifier les champs \enquote{BUTTON\_WIDTH} et \enquote{BUTTON\_HEIGHT} du fichier $.yaml$.

%--------------------------
\section{Ressources xcf}
Il s'agit de ressources dépendantes d'autres ressources : écrans d'aide et tutoriels. Par exemple, le menu d'aide concernant le bloc simple devra être modifié si l'image du bloc simple est modifiée. 

%---------------------------
\section{Attention : Compatibilité mobile}
Pour que vos modifications soient effectives sur les applications mobiles, il faut recompiler toute l'application. Pour ce faire, référez-vous à la documentation technique.

\end{document}

