\documentclass[11pt]{article}
\usepackage[utf8]{inputenc}
\setlength{\topmargin}{-.5in}
\setlength{\textheight}{9in}
\setlength{\oddsidemargin}{.125in}
\setlength{\textwidth}{6.25in}
\begin{document}
\title{Documentation Développeur}
\author{Thomas Bombrun, Dimitri Bouleau, Thomas Coeffic, Caroline Ramond, Théo Van-Lede\\
INPG - Ensimag}
\renewcommand{\today}{17 Juin 2014}
\maketitle
Ce document est un complément de la documentation générée automatiquement, 
qui justifie l'implémentation et présente la structure des dossiers.

\section {Javascript}
Ces fichiers sont stockés dans le dossier {\em js}.

\section {Ressources}
Le dossier {\em ressources} contient les fichiers png chargées par le jeu,
elles sont définies dans {\em conf.yaml}.\\
Le dossier {\em ressource-dev} contient les fichiers gimp correspondant à
ces fichiers png, et peuvent être utilisés pour modifier plus rapidement 
les ressources.\\
Les sons associés aux actions sont stockés dans {\em ressources/sounds}, afin 
d'assurer une compatibilé avec firefox ne pas utiliser de fichier mp3 (ogg
semble une bonne alternative).\\

\section {Justification de l'implémentation}
\begin{enumerate}
	\item
		\textbf{Cookie.js} :\\
		On utilise {\em constants.USE\_CORDOVA} pour savoir si on 
		utilise les cookies stars et levelMax, ou si on utilise
		la clé stars pour stocker ces données.\\
		Cela permet d'avoir un code unique pour la plateforme mobile et
		web.\\
	\item
		\textbf{eventFunction.js} :\\
		Dans toutes les fonctions appelées par overlap (défini dans 
		phaser.physics.arcade), on renforce la condition en forçant les
		deux sprites à être au moins à moitié chevauchante pour
		effectuer l'action. Cela permet d'éviter les bugs liés à la
		bille déclenchant deux overlap simultanés.\\
	\item
		\textbf{main.js} : \\
		On utilise une fonction pour trouver la valeur correspondant à
		un retour 'OK' après une requête de type 
		{\em XMLHTTPRequest.open}, cela permet d'avoir un code
		fonctionnant pour Android (OK = 200, erreur = 0), iOS (OK = 0,
		erreur = 404) et Desktop (OK = 200, erreur = 404) fonctionnel, 
		sans utiliser user-agent qui est peu fiable. Cependant cette
		approche nécéssite de rattraper l'erreur dans le cas où le 
		niveau test n'est pas accessible (et donc renvoie une mauvaise 
		valeur pour OK), pour cela il faut définir une limite maximale
		pour le nombre de niveau.\\
	\item
		\textbf{moveball.js} : \\
		Ce fichier contient toute la gestion des déplacement de la
		bille, la gestion du score et la gestion des blocs {\em turn}.\\
		Les blocs {\em turn} sont très sensibles aux lags, il faut
		donc prévoir beaucoup de rattrapage d'erreurs liées à des 
		mauvaises positions.\\
		Pour le score il y a simplement {\em checkMoveGroup} qui
		vérifie la validité du déplacement demandé et permet donc de 
		créer .\\
\end{enumerate}
\end{document}
