\documentclass[11pt]{article}
\usepackage[utf8]{inputenc}
\setlength{\topmargin}{-.5in}
\setlength{\textheight}{9in}
\setlength{\oddsidemargin}{.125in}
\setlength{\textwidth}{6.25in}
\begin{document}
\title{Documentation Développeur}
\author{Thomas Bombrun, Dimitri Bouleau, Thomas Coeffic, Caroline Ramond, Théo Van-Lede\\
INPG - Ensimag}
\renewcommand{\today}{17 Juin 2014}
\maketitle
Ce document est un complément de la documentation générée automatiquement, 
qui justifie l'implémentation et présente la structure des dossiers.

\section {Javascript}
Ces fichiers sont stockés dans le dossier {\em js}.
\begin{itemize}
	\item
		\textbf{constants.js} : Permet de charger le fichier de 
		configuration yaml.\\
	\item
		\textbf{cookie.js} : Assure la gestion des cookies (dans le
		cas d'une utilisation web) ou de la mémoire locale (dans 
		le cas d'une utilisation mobile), afin de conserver l'état 
		d'avancement de l'utilisateur dans les niveaux.\\
	\item
		\textbf{create.js} : Permet de créer les différents écrans du
		jeu : le menu principal, le menu de séléction de niveau, et 
		l'écran correspondant à un début de partie.\\
	\item
		\textbf{createLevel.js} : Met à jour l'index des niveaux, 
		fonctionne pour le tutoriel et le jeu classique.\\
	\item
		\textbf{eventFunction.js} : Gère la majorité des événements 
		associés aux clicks dans les menus et en jeu, et aussi les 
		actions déclenchées durant une partie (affichage de l'écran de 
		fin, rencontre avec un bloc).\\
	\item
		\textbf{help.js} : Gère les écrans d'aide.\\
	\item
		\textbf{main.js} : Charge les différentes ressources 
		nécéssaires, contient la définition des variables globales 
		utilisées dans plusieurs fichiers.\\
	\item
		\textbf{moveball.js} : Gère les déplacements de la bille 
		(prise en compte des actions de l'utilisateur, appel des 
		fonctions adéquates dans {\em eventFunction} en cas de 
		collision/chevauchement.\\
		Gère aussi les déplacements et repositionnements nécéssaires 
		pour les blocs {\em turn}.\\
	\item
		\textbf{parser.js} : Permet la lecture et le chargement en 
		mémoire d'un fichier de niveau.\\
	\item
		\textbf{sound.js} : Gère les animations sonores du jeu.\\
\end{itemize}
\section {Ressources}
Le dossier {\em ressources} contient les fichiers png chargées par le jeu,
ils sont définis dans {\em conf.yaml}.\\
Le dossier {\em ressource-dev} contient les fichiers gimp correspondant à
ces fichiers png, et peuvent être utilisés pour modifier plus rapidement 
les ressources.\\
Les sons associés aux actions sont stockés dans {\em ressources/sounds}, afin 
d'assurer une compatibilé avec firefox ne pas utiliser de fichier mp3 (ogg
semble une bonne alternative).\\
Les niveaux sont eux dans le dossier levels, ils doivent être de la forme 
"\%nombre.txt" et se suivre. Il est en de même pour les niveaux du tutoriel 
qui sont dans le dossier tutorial.\\

\section {Justification de l'implémentation}
\begin{itemize}
	\item
		\textbf{Cookie.js} :\\
		On utilise {\em constants.USE\_CORDOVA} pour savoir si on 
		utilise les cookies stars et levelMax, ou si on utilise
		la clé stars pour stocker ces données.\\
		Cela permet d'avoir un code unique pour la plateforme mobile et
		web.\\
	\item
		\textbf{eventFunction.js} :\\
		Dans toutes les fonctions appelées par overlap (défini dans 
		phaser.physics.arcade), on renforce la condition en forçant les
		deux sprites à être au moins à moitié chevauchante pour
		effectuer l'action. Cela permet d'éviter les bugs liés à la
		bille déclenchant deux overlap simultanés.\\
	\item
		\textbf{main.js} : \\
		On utilise une fonction pour trouver la valeur correspondant à
		un retour 'OK' après une requête de type 
		{\em XMLHTTPRequest.open}, cela permet d'avoir un code
		fonctionnant pour Android (OK = 200, erreur = 0), iOS (OK = 0,
		erreur = 404) et Desktop (OK = 200, erreur = 404) fonctionnel, 
		sans utiliser user-agent qui est peu fiable. Cependant cette
		approche nécéssite de rattraper l'erreur dans le cas où le 
		niveau test n'est pas accessible (et donc renvoie une mauvaise 
		valeur pour OK), pour cela il faut définir une limite maximale
		pour le nombre de niveau.\\
	\item
		\textbf{moveball.js} : \\
		Ce fichier contient toute la gestion des déplacement de la
		bille, la gestion du score et la gestion des blocs {\em turn}.\\
		Les blocs {\em turn} sont très sensibles aux lags, il faut
		donc prévoir beaucoup de rattrapage d'erreurs liées à des 
		mauvaises positions.\\
		Pour le score il y a simplement {\em checkMoveGroup} qui
		vérifie la validité du déplacement demandé et permet donc de 
		créer .\\
\end{itemize}
\end{document}
